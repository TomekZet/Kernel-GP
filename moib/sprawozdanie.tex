\documentclass{article}
\usepackage{polski} %może wymagac dokonfigurowania latexa, ale jest lepszy niż standardowy babel'owy [polish]
\usepackage[utf8]{inputenc}
\usepackage[OT4]{fontenc}
\usepackage{amsfonts}
\usepackage{graphicx,color} %include pdf's (and png's for raster graphics... avoid raster graphics!)
\usepackage{url}
\usepackage[pdftex,hyperfootnotes=false,pdfborder={0 0 0}]{hyperref} %za wszystkimi pakietami; pdfborder nie wszedzie tak samo zaimplementowane bo specyfikacja nieprecyzyjna; pod miktex'em po prostu nie widac wtedy ramek


% Zmiana rozmiarów strony tekstu
\addtolength{\voffset}{-1cm}
\addtolength{\hoffset}{-1cm}
\addtolength{\textwidth}{2cm}
\addtolength{\textheight}{2cm}

%bardziej zyciowe parametry sterujace rozmieszczeniem rysunkow
\renewcommand{\topfraction}{.85}
\renewcommand{\bottomfraction}{.7}
\renewcommand{\textfraction}{.15}
\renewcommand{\floatpagefraction}{.66}
\renewcommand{\dbltopfraction}{.66}
\renewcommand{\dblfloatpagefraction}{.66}

\newcommand{\termdef}[1]{\emph{#1}}
\newcommand{\acronym}[1]{\emph{\MakeUppercase{#1}}}

\newcommand{\definicja}[1]{\termdef{#1}}
\newcommand{\akronim}[1]{\acronym{#1}}
\newcommand{\english}[1]{ang.~\emph{#1}}

\setcounter{topnumber}{9}
\setcounter{bottomnumber}{9}
\setcounter{totalnumber}{20}
\setcounter{dbltopnumber}{9}

% własny bullet list z malymi odstepami
\newenvironment{tightlist}{
\begin{itemize}
  \setlength{\itemsep}{1pt}
  \setlength{\parskip}{0pt}
  \setlength{\parsep}{0pt}}
{\end{itemize}}




\begin{document}

\thispagestyle{empty} %bez numeru strony

\begin{center}
{\large{Sprawozdanie z eksperymentu z przedmiotu\\
Metaheurystyki i Obliczenia Inspirowane Biologicznie}}

\vspace{3ex}



\vspace{3ex}
{\footnotesize\today}

\end{center}

\vspace{10ex}

Prowadzący: dr inż. Maciej Komosiński

\vspace{5ex}

Autorzy:
\begin{tabular}{lllr}
\textbf{Tomasz Ziętkiewicz} & inf84914 & ISWD & tomek.zietkiewicz@gmail.com \\
\end{tabular}

\vspace{5ex}

Zajęcia poniedziałkowe, 15:10.

\newpage


\begin{abstract}
\emph{SVM} 
\end{abstract}


\section{Wstęp}
	\subsection{Opis eksperymentu}
	Celem eksperymentu było użycie programowania genetycznego do wyewoluowania optymalnych funkcji jądrowych dla klasyfikatora \emph{SVM}.
	Klasyfikator SVM dokonuje klasyfikacji binarnej oddzielając od siebie dwie grupy przykładów hiperpłaszczyzną przebiegającą w przestrzeni atrybutów opisujących przykłady. Najczęściej grupy te nie są liniowo separowalne i trzeba dokonać transformacji cech opisujących przykłady do przestrzeni o większej liczbie wymiarów tak, żeby były w niej liniowo separowalne. Funkcje używane do dokonania tej transformacji to funkcje jądrowe  (inczej kernele  - ang. kernel functions). Wybór odpowiedniej funkcji zależy od rozwiązywanego problemu i zazwyczaj opiera się na doświadczeniu osoby używającej klasyfikator, posiadanej przez nią wiedzy dziedzinowej. W przypadku nie znanych apriori danych wejściowych i braku doświadczenia w wyborze optymalnej funkcji jądrowej można posłużyć się automatycznymi metodami optymalizacji. Wśród nich idalną do tego zadania wydaje się \definicja{programowanie geetyczne} (\akronim{GP} - \english{Genetic Programming}). Jest to szczególny rodzaj \definicja{obliczeń ewolucyjnych} (\akronim{EC} - \english{Evolutionary Computing}), w którym ewoluowane osobniki to funkcje reprezentowane za pomocą struktur drzewiastych.
	Na potrzeby eksperymentu zaprojektowano proces ewolucyjny, który ewoluował funkcje jądrowe poprzez łączenie ze sobą prostych funkcji jądrowych w funkcje złożone za pomocą funkcji łączących.
	
	
	
	

%			\begin{figure}[h]
%				\includegraphics[scale=0.90]{../results/quality}
%				\caption{Porównanie średniej jakości rozwiązań generowanych przez algorytmy dla różnych instancji \emph{QAP}\label{fig:quality}}
%			\end{figure}


		
		
\section{Podsumowanie}
	\subsection{Wnioski}
		\begin{itemize}
			\item 
		\end{itemize}


%%%%%%%%%%%%%%%% literatura %%%%%%%%%%%%%%%%

%\bibliography{sprawozdanie}
%\bibliographystyle{plain}

\end{document}

