\chapter{Case study - klasyfikacja danych ADHD 200}
\definicja{ADHD 200} był międzynarodowym konkursem, który zakończył się we wrześniu 2011 roku. Dzięki współpracy ośmiu szpitali i ośrodków naukowych z całego świata w ramach konkursu udostępniono zbiór zawierające dane medyczne 776 dzieci, z czego 285 z ADHD. Zadaniem uczestników konkursu było skonstruowanie klasyfikatora diagnozującego ADHD na podstawie tych danych. Zbiór testowy zawierał dane xxx dzieci, których danych nie było w zbiorze uczącym, bez podanej diagnozy. Celem skonstruowanego klasyfikatora było przypisanie diagnozy do przykładów ze zbioru testującego.

Wyniki konkursu pokazały, że zbiór danych był trudny w klasyfikacji. Największa osiągnięta trafność klasyfikacji wyniosła $ 60.51 \% $ (szczegółowe wyniki dostępne na stronie konkursu: [\ref{url:adhd200}].
\section{Opis zbioru danych}
	\subsection{Surowe dane}
	Dane dostarczone przez organizatorów konkursu składają się z:
	\begin{itemize}
		\item Danych klinicznych:
			\begin{itemize}
				\item Płeć
				\item Wiek
				\item Współczynnik IQ
				\item Prawo/lewo ręczność
			\end{itemize}
		\item Danych obrazowych:
			\begin{itemize}
				\item Strukturalnych - dane pochodzące z obrazowania rezonansu magnetycznego (\akronim{MRI}, \english{Magnetic Resonance Imaging}). Są to trójwymiarowe obrazy o rozdzielczości ok. 256x254x160 punktów, obrazujące strukturę mózgu osoby badanej
				\item Funkcjonalnych - dane pochodzące z obrazowania funkcjonalnego rezonansu magnetycznego (\akronim{fMRI}, \english{functional Magnetic Resonance Imaging}) będące sekwencją ok 120 trójwymiarowych obrazów o rozdzielczości ok 250x250x250 punktów, obrazującą aktywność mózgu osoby badanej rejestrowaną przez ok 6 minut.
			\end{itemize}
	\end{itemize}
	\subsection{Preprocessing}

\section{Konstrukcja i selekcja cech}
\section{Wyniki klasyfikacji}
	\subsection{Kernel GP}
	\subsection{Porównanie z innymi algorytmami}
		\subsubsection*{SVM}
		\subsubsection*{Inne klasyfikatory}
