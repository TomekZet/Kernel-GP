\chapter{Wprowadzenie}
Jako cechę wyróżniającą nasz gatunek, \textit{Homo sapiens sapiens},  podaje się zwykle umiejętność inteligentnego myślenia oraz wytwarzania i używania narzędzi. Korzystając zarówno z inteligencji jak i wytworzonych wcześniej narzędzi, człowiek przez tysiąclecia rozwoju kultury i cywilizacji tworzył coraz to nowsze i doskonalsze narzędzia, które ułatwiały mu wykonywanie trudnych i monotonnych prac, a czasami wykonywały te prace za niego. Ten proces udoskonalania wytworów człowieka, zbliżony jest do procesu ewolucji naturalnej, której podlega człowiek, dlatego rozwój człowieka jako gatunku opisuje się często jako złożenie ewolucji naturalnej i kulturowej. Narzędzia, które sprawdzają się lepiej od już istniejących zastępują je, rozwiązania krzyżują się, te powstałe na potrzeby jednej dziedziny są stosowane w innych dziedzinach.
Narzędzia, których używa człowieka, mogą być postrzegane jako jego "rozszerzenie" czy "przedłużenie" (\english{extension}) \cite{ANAL:ANAL096}. Na przykład tak prosty przyrząd jak łopata może być uznany za "przedłużenie" rąk i dłoni. Inne narzędzia mogą stanowić rozszerzenie ludzkiego umysłu: notatnik stanowi rozszerzenie ludzkiej pamięci pozwalając odciążyć ją od konieczności przechowywania pewnych informacji. O ile rozwój narzędzi stanowiących rozszerzenie ludzkich zdolności fizycznych ma miejsce od początków dziejów, o tyle rozszerzenia umysłu przez tysiąclecia zmieniały się tylko kilka razy. Za największy wynalazek zwiększający nasze możliwości intelektualne należy uznać rozwój języka, pisma i druku. Poza nim tworzone przez tysiąclecia narzędzia nie przynosiły rewolucyjnych zmian. Dopiero wynalazki powstałe w XIX wieku, takie jak telegraf, telefon, radio zapoczątkowały rewolucję w łatwości z jaką człowiek zdobywał i przekazywał informacje. Rewolucja ta osiągnęła swój pełen rozkwit wraz z pojawieniem i rozwojem komputerów. Są one rozszerzeniem umysłu, które wspomaga człowieka w najbardziej złożonych i skomplikowanych operacjach mentalnych. Co więcej nie stanowią jedynie przedłużenia pojedynczej jego części, ale umysłu jako całego. O związku łączącym komputer i umysł świadczy nawet etymologia słowa "komputer" - w języku angielskim początkowo oznaczało ono "osobę, która liczy" (\english{one who computes}), wykonuje obliczenia matematyczne. W języku polskim z kolei istniało określenie "mózg elektronowy" oznaczające komputer.

Komputer stanowi rozszerzenie ludzkiego umysłu między innymi poprzez automatyzację procesu przetwarzania informacji - zwalnia człowieka z konieczności wykonywania monotonnych, powtarzających się czynności umysłowych. Dzięki możliwości operowania na większej liczbie elementów na raz oraz szybszym tempie wykonywania prostych operacji komputer umożliwia również wykonywanie zadań, których człowiek bez jego pomocy w ogóle nie byłby w stanie wykonać. Użycie komputera w celu rozwiązania jakiegoś problemu wymaga jednak od człowieka jego zdefiniowania w taki sposób, żeby komputer mógł je w jednoznaczny sposób przełożyć na ciąg prostych operacji matematycznych wykonywanych na danych. Taka definicja sposobu działania komputera, które prowadzi do rozwiązania przez niego pewnego zadania to program komputerowy. W przypadku współczesnych komputerów każdy program musi ostatecznie sprowadzać się do ciągu instrukcji, które mają być po kolei wykonywane. Najbardziej bezpośrednim sposobem zapisu programu jest bezpośrednie podanie ciągu takich instrukcji, na przykład w języku asemblera, który odnosi się do elementarnych operacji wykonywanych przez komputer. Używanie takiej formy definicji zadania jest jednak bardzo niewygodne dla człowieka. Dlatego istnieją języki programowania o różnym stopniu abstrakcji - od języków niskiego poziomu, zawierających proste operacje, związane ściśle ze sposobem pracy komputera, po języki wysokiego poziomu, abstrahujące od procesów przebiegających w komputerze i operujące na wyrażeniach odnoszących się do w sposób bardziej bezpośredni do rozwiązywanego problemu.

Mimo używania języków wysokiego poziomu, człowiek wciąż musi wykonać sporo pracy, żeby przełożyć swoje potrzeby na język zrozumiały dla komputera. W pewnym sensie komputer przypomina niedoświadczonego, niezbyt inteligentnego pracownika, którego przełożony musi krok po kroku instruować, nieraz spędzając przy tym więcej czasu niż zajęłoby mu wykonania danego zadania samodzielnie. Na szczęście zarówno w przypadku komputera jak i niedoświadczonego pracownika trud włożony w napisanie programu / wytłumaczenie pracownikowi sposobu wykonywania zadania opłaca się, jeśli tylko program/zadanie musi być wykonywane wiele razy. Problem pojawia się wtedy, gdy zmianie ulegają pewne warunki zadania, na przykład dane, na których zdanie jest wykonywane - wymagają one elastyczności. Elastyczność programu komputerowego może być w pewnym stopniu zapewniona przez parametry, którymi można regulować jego pracę - ich odpowiednie ustawienie wymaga jednak inteligencji osoby uruchamiającej program. W przypadku pracownika, jeśli jeśli nie wykaże się on wystarczającą elastycznością i nie poradzi sobie z problemem po jego nieznacznym zmodyfikowaniu przełożony może stwierdzić, że nie powinien zajmować się powierzonymi mu problemami, ponieważ przy każdej modyfikacji zadania potrzebna jest ingerencja osób bardziej od niego doświadczonych. Taka elastyczność w rozwiązywaniu problemów, wykonywaniu zadań w zmieniających się warunkach jest często uważana za jeden z przejawów inteligencji. Programy, które wykazują się elastycznością na zmieniające się dane, potrafią same dostosowywać swoje parametry, również zwane są inteligentnymi. Dziedzina informatyki zajmująca się między innymi takimi programami zwana jest Sztuczną inteligencją.

Umiejętność uczenia się podawana jest podawana jako jeden z wyznaczników inteligencji. Systemy uczące się, to systemy (np. programy komputerowe), które potrafią doskonalić sposób swojego działania wraz z nabywanym doświadczeniem. Dziedzina sztucznej inteligencji zajmująca się ich konstruowaniem to \emph{uczenie maszynowe} (\english{Machine Learning}). Systemy takie pomagają uwolnić człowieka nie tylko od wykonywania szeregu 



, uczenie maszynowe, konieczność ustawiania Parametrów

Algorytmy ewolucyjne

Człowiek tworzy narzędzia, te modyfikują się same, tworzą inne narzędzia, korzystają z narzędzi.



\section{Cel i~zakres pracy}
Niniejsza praca ma dwa podstawowe cele:
\begin{itemize}
\item Stworzenie algorytmu programowania genetycznego optymalizującego parametry klasyfikatora SVM
%\item Zastosowanie stworzonego algorytmu do klasyfikacji danych ze zbioru ADHD-200
\end{itemize}
%Realizacja drugiego z powyższych celów służyć ma przede wszystkim sprawdzeniu efektywności stworzonego algorytmu, ale jest też wyzwaniem samym w sobie. Zbiór danych ADHD-200 nie poddaje się łatwo klasyfikacji za pomocą metod uczenia maszynowego, dlatego każda poprawa wyników klasyfikacji względem wyników dotychczas osiąganych będzie sporym sukcesem.

\section{Struktura pracy}
Struktura pracy jest następująca: rozdział drugi przedstawia ważniejsze zagadnienia teoretyczne związane z pracą oraz zawiera przegląd literatury. W rozdziale trzecim opisano zaimplementowany algorytm Kernel GP oraz przedstawiono sposób jego implementacji. Rozdział czwarty przedstawia wyniki działania algorytmu na standardowych zbiorach danych używanych do testowania algorytmów maszynowego uczenia. W rozdziale piątym prezentowane są wyniki działania algorytmu na zbiorze ADHD-200. Rozdział szósty zawiera podsumowanie.


