\chapter{Wprowadzenie}


\section{Cel i~zakres pracy}
Niniejsza praca ma dwa podstawowe cele:
\begin{itemize}
\item Stworzenie algorytmu programowania genetycznego optymalizującego parametry klasyfikatora SVM
\item Zastosowanie stworzonego algorytmu do klasyfikacji danych ze zbioru ADHD-200
\end{itemize}
Realizacja drugiego z powyższych celów służyć ma przede wszystkim sprawdzeniu efektywności stworzonego algorytmu, ale jest też wyzwaniem samym w sobie. Zbiór danych ADHD-200 nie poddaje się łatwo klasyfikacji za pomocą metod uczenia maszynowego, dlatego każda poprawa wyników klasyfikacji względem wyników dotychczas osiąganych będzie sporym sukcesem.

\section{Struktura pracy}
Struktura pracy jest następująca: rozdział drugi przedstawia ważniejsze zagadnienia teoretyczne związane z pracą oraz zawiera przegląd literatury. W rozdziale trzecim opisano zaimplementowany algorytm Kernel GP oraz przedstawiono sposób jego implementacji. Rozdział czwarty przedstawia wyniki działania algorytmu na standardowych zbiorach danych używanych do testowania algorytmów maszynowego uczenia. W rozdziale piątym prezentowane są wyniki działania algorytmu na zbiorze ADHD-200. Rozdział szósty zawiera podsumowanie.


